The twentieth century for physics was marked by the successful theories of quantum mechanics and the theory of general relativity. However, a unification of these two theories has not yet been achieved and is one of the biggest challenges in modern physics. To test the quantum nature of gravity, \citeauthor{bose_spin_2017} proposed an experiment to entangle two massive particles through gravity\cite{bose_spin_2017}. This thesis is a step towards this experiment and builds on our previous work \cite{janse_characterization_2024,eli,mart}. We demonstrate stable levitation of a \ce{NdFeB} particle with a diameter of $\qty{12}{\micro\meter}$ in an on-chip planar magnetic Paul trap. Levitation was observed at atmospheric pressure all the way down to $\qty{1E-4}{\milli\bar}$. At atmospheric pressure we observed the $x$, $y$, $\gamma$ and $\beta$ modes, but with very low Q-factors ($Q \approx 5$). At lower pressures the Q-factor increases ($Q \approx 3000$), and becomes independent of the chamber pressure. The on-chip design opens the possibility to integrate the trap with NV centres to ground state cool the particle. Our results suggest that this method allows the trapping of a \qty{1}{\micro\meter} particle.
