\chapter{Conclusion}
\label{chap:conclusion}
In this thesis we have shown successful levitation of a \qty{12}{\micro\meter} sized \ce{NdFeB} particle in a planar magnetic Paul trap. The trap was fabricated using a combination of nanofabrication techniques. Levitation was observed at atmospheric pressure all the way down to \qty{1E-4}{\milli\bar}. At atmospheric pressure we succesfully observed the $x$/$y$ and $\gamma$/$\tilde\beta$ modes. Due to the low Q-factors at atmospheric pressure it was not possible to tell the difference between the $x$ and $y$ mode or the $\gamma$ and $\beta$ mode. The dependence of $\omega_{x,y}$ on $i_1$ and $\Omega$ follows the expected relation from theory.

At lower pressures the Q-factors increase untill we reach roughly \qty{1E-2}{\milli\bar} where the Q-factor tends to a constant value. We attribute this to Eddy current damping, though we do not know where it occurs exactly. Further more careful measurements of the Q-factors at low pressures are needed to determine the origin of the damping.

When measuring using a laser at low pressures we observed a loss of magnetization. A future project will work on interferometric readout, which allows us to use a lower laser intensity. We also suspect that the levitation height is very low, because of this we think that we do not need a hole in the top cover glass. In the future this will help us to use diamonds with NV centers to couple to the particle more easily.
TODO: Readout or also sideband cooling
- Many spins to cool
- Many spins to change angular momentum
- Single spin for stern gerlach experiment

- smaller particle
- elongated particle (Huellery, Gabriel etet ferromagnet diamond)
- hoger B0 veld (met kern)
- magnetisatie (Milan)
