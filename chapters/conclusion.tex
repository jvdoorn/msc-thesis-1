\chapter{Conclusion and outlook}
\label{chap:conclusion}
In this thesis we have shown successful levitation of a \ce{NdFeB} particle with a \qty{12}{\micro\meter} diameter in a planar magnetic Paul trap. The trap was fabricated using a combination of nanofabrication techniques. Levitation was observed at atmospheric pressure all the way down to \qty{1E-4}{\milli\bar}. At atmospheric pressure we succesfully observed the $x$, $y$, $\gamma$ and $\beta$ modes. Due to the low Q-factors at atmospheric pressure it was not possible to tell the difference between the $x$ and $y$ mode or the $\gamma$ and $\beta$ modes. We furthermore observed the expected relations $\omega_{x,y} \propto i_1 \propto 1/\Omega$ and $\omega_{\gamma,\beta} \propto \sqrt{B_0}$.

At lower pressures the Q-factors increase untill we reach roughly \qty{1E-2}{\milli\bar} where the Q-factors tend to a constant value. We think this is due to Eddy current damping, though we have been unable to reproduce it in simulations. More detailed measurements of the Q-factors at low pressures are needed to determine the origin of the damping. At low pressures we also observed the \zmode. Again at low pressures we saw the expected relation between $\omega_{x,y,z} \propto 1/\Omega$.

The direct gaps in our knowledge are: the dependence of the Q-factor on pressure for the $z$, $\gamma$ and $\beta$-mode; the dependence of $\omega_z$ on $B_0$; the dependence of $z_0$ on $B_2'$; and the origin of the damping at low pressures. We note that the damping of the \zmode is statistically significantly higher than the damping of the \xmode and \ymode. This might shed light on the origin of the damping, but further investigation is needed. We suggest performing a time dependent simulation in COMSOL and properly model potential sources of damping such as eddy currents inside the particle or tracks.

When measuring using a laser at low pressures we observed a loss of magnetization. A future project will work on interferometric readout, which allows us to use a lower laser intensity. This will enable us to fill most of our knowledge gaps about the parameter dependences. In addition to this we are also looking to increase the remnant magnetization of the particle and to reach a higher $\vec{B_0}$ field by adding a core to the Helmholtz coils. In addition to this we are also looking to trap a smaller (\qty{1}{\micro\meter} diameter) particle as a step towards the quantum regime. We expect that we will be able to trap this particle using the same approach as we used for the \qty{12}{\micro\meter} particle.

Even more long term we are looking to use NV centers. If we replace the cover glass with a diamond we can use the NV centers as a readout. Due to the movement of the particle the emission of the NV centers will split in two bands due to the Zeeman splitting of the $\ket{+1}$ and $\ket{-1}$ states of the NV centers. An additional use of the NV centers is to cool the particle using sideband cooling, similar to the work of \textcite{delord_spin-cooling_2020}. A key step in this case are high eigenfrequencies and good Q-factors in order to reach a sideband resolved regime. The idea is to use the rotational modes (which have an order of magnitude of \qty{1}{\kilo\hertz}) to cool the particle. These rotational modes can be `boosted' by using an elongated particle\cite{huillery_spin-mechanics_2020}.
