\chapter{Introduction}
The Hensen Lab aims to understand the interplay of quantum mechanics and gravity. Recently a experiment was proposed by \citeauthor{bose_spin_2017} to probe the quantum mechanical nature of gravity. The central idea is to entangle two particles through gravity which is only possible if gravity is a quantum entity. The idea is to levitate two small ($\approx \qty{1}{\micro\meter}$) particles and cool them to their ground state. The particles are dropped through a Stern-Gerlach interferometer and the entanglement is measured by the interference pattern.

Currently, we as a group, are working towards the levitation of micrometer sized particles. Previous projects within our group worked on the levitation of a \qty{100}{\micro\meter} sized particle in a planar magnetic Paul Trap\cite{eli, mart}. This trap was realised on PCB and the goal of this project is to miniturize the trap. Miniturization is key to reach the quantum regime. Furthermore a on-chip trap enables easier integration with other components such as NV centers.
