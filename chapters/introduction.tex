\chapter{Introduction}
\label{chap:introduction}
The twentieth century brought us two major cornerstones in physics: quantum mechanics and general relativity. Whilst quantum mechanics successfully describes physics at the smallest scales and general relativity describes physics at the largest scales, a unified theory of quantum gravity is still missing. To test whether gravity is a quantum entity (the existence of gravitons) an experiment was proposed by \textcite{bose_spin_2017}.

\citeauthor{bose_spin_2017} suggest entangling two particles through gravity\cite{bose_spin_2017}. If this is possible, it would be a direct proof that gravity is a quantum entity. This experiment requires two cat states. A cat state is a superposition of a macroscopic object. To achieve this we need to cool two objects of roughly \qty{1}{\micro\meter} to their ground state. The superpositions are created using a Stern-Gerlach interferometer. The particles are then allowed to interact through gravity. The outcome of the experiment is measured by the interference of the particles.

In this thesis we focus on the levitation of micrometer sized particles using a planar magnetic Paul trap. A magnetic Paul trap uses an ac magnetic field to obtain stable levitation. This builds on previous work within our group where a \qty{250}{\micro\meter} cubic \ce{NdFeB} magnet was levitated in a planar magnetic Paul trap realized on a PCB\cite{eli, mart}. In order to work towards the quantum regime, the system needs to be miniaturized.

Two other well known levitation techniques are optical (eigenfrequencies between \qtyrange{10}{300}{\kilo\hertz}) and electrical traps (eigenfrequencies between \qtyrange{1}{10}{\kilo\hertz})\cite{levitodynamics}. Compared to these techniques, magnetic traps have the disadvantage of having a lower eigenfrequency (\qtyrange{1}{10}{\kilo\hertz}), often require cryogenic temperatures and no on-chip integration exists\cite{levitodynamics}. The advantage however is that magnetic traps have the potential to levitate much larger and heavier particles. Levitating heavier particles creates a strong gravitational interaction which is beneficial for the experiment.

The cryogenic necessity originates from the fact that many magnetic traps use the Meissner effect to levitate a particle. An alternative to this however is to use non-static magnetic fields which then also satisfies Earnshaw's theorem. Previous work has shown this to be possible\cite{perdriat,eli,mart}. The goal of this project is to create an on-chip variant of the magnetic Paul trap. Besides the advantage of an on-chip design, this will also increase the eigenfrequency of the trap\cite{perdriat}.
