\chapter{Discussion}
\label{chap:discussion}

\section*{Loss of levitation}
A major source of confusion (and frustration) was the sudden loss of levitation. For long periods of time our particle was easily trapped without major issues. However, just before the Christmas holiday it became very difficult to trap. We had previously seen that heating can cause the particle to loose its magnetization, but didn't appear to be the case as it still responded to the magnetic field. An attempt was mode to remagnetize the particle which did not matter much. Using an optical microscope we noted a black `smut' in and around the trap. This was further confirmed using SEM measurements. This `smut' was not found on earlier images before we used the trap. See \autoref{fig:smut-optical-microscope}.

\begin{SCfigure}[50]
    \centering
    \includegraphics[width=0.4\textwidth]{figures/sample/dirt_optical_microscope.jpeg}
    \caption{An optical microscope image showing the `smut' around the trap. This `smut' is also present inside of the trap.}
    \label{fig:smut-optical-microscope}
\end{SCfigure}

The current hypothesis is that this `smut' limits the free movement of the particle. This could be the case if the particle has a very low levitation height. The origin of the `smut' is likely due to oil from the roughing pump. Due to the heat of the Helmholtz coils (roughly \qty{60}{\celsius}) we think the oil evaporated or burnt and was deposited around the trap. The oil hypothesis is strengthened by the fact that we found an oil like substance on the glass of the vacuum chamber. At the time of writing the group is working on creating a new sample.

Due to the loss of levitation we were also unable to further study the dependence of the \zmode on other parameters or its Q-factor at low pressures.

\section*{Levitation height}
Changing the focus of the objective used to image the particle lets us estimate the levitation height of the particle. Based on this we noticed that the particle appears to levitate very close to the bottom of the trap. Furthermore, by changing the gradient field ($\vec{B_2}$) we did not notice any change. We think that the trap stiffness in $z$ is relatively much larger than the PCB version of the trap. A low levitation height also strengthens the hypothesis that the `smut' hindered the movement of the particle.

\section*{Loss of magnetization}
As mentioned in \autoref{chap:results}, the particle lost its magnetization when irradiated with the laser at low pressures ($<\qty{1}{\milli\bar}$). Decreasing the laser intensity also ment a smaller SNR. Due to this tradeoff a decision was made to only use camera measurements at low pressure. A comparision between the two methods will follow. Besides the total loss of magnetization we are also not sure how well the particle retains its magnetization over time. In addition to this we are also not sure what the resulting magnetization is since we do not reach the saturation field. A future project will focus on designing a device that can fully saturate the particle.

\section*{Damping at low pressures}
We observed a limit in the Q-factor for pressures below \qty{1E-2}{\milli\bar}. The origin of this is likely Eddy currents. Our attempts to model this however have been unsuccessful, as can be seen in \autoref{tab:dissipation}. Another possibility is that the Eddy currents exist inside of the particle, something which we did not model. Furthermore, other sources of damping should be considered such as noise from the electronics (or other Brownian noise sources) or anisotropy in the particle\cite{millen}. The latter case might be influenced by the static field $\vec{B_0}$, it could be interesting to study the dependence of the dissipation on the static field.

\section*{Laser v.s. camera readout}
\begin{tabularx}{\textwidth}{XX}
    \toprule
    Camera readout & Laser readout \\
    \midrule
    \begin{itemize}[left=0pt,topsep=0pt,label=\textcolor{green}{\texttt{+}}]
        \item Immediate visual feedback about the behaviour of the particle;
        \item SNR only dependent on the contrast of the particle (which is really good with a flat field correction);
    \end{itemize} \begin{itemize}[left=0pt,topsep=0pt,label=\textcolor{red}{\texttt{-}}]
        \item Limited sample rate (roughly \qty{400}{\fps});
        \item No `real' lock-in measurements possible;
        \item Analysis is not directly possible;
    \end{itemize} & \begin{itemize}[left=0pt,topsep=0pt,label=\textcolor{green}{\texttt{+}}]
        \item High sample rate that is only limited by the gain bandwidth product of the photodiode;
        \item `Real' lock-in measurements possible by connecting the photodiode to a lock-in amplifier;
    \end{itemize} \begin{itemize}[left=0pt,topsep=0pt,label=\textcolor{red}{\texttt{-}}]
        \item SNR is dependent on the laser intensity, background light and electronic noise;
        \item No immediate visual feedback about the behaviour of the particle;
        \item High laser intensity can cause the particle to loose its magnetization;
    \end{itemize} \\
    \bottomrule
\end{tabularx}

In the future laser readout is key to properly study the $\gamma$ and $\beta$ modes at low pressures. The idea is to move to an interferometric setup to measure the position of the particle. This has a higher SNR if done correctly. In addition to this we are also investigating the use of NV centers in diamond to measure the position of the particle. An additional advantage of NV centers is that they will also allow for sideband cooling.

\section*{Lorentzian fits}
To obtain the data in \autoref{fig:xyz-mode-dependence-on-trapping-frequency-1mbar} we fitted the peaks in our data with a Lorentzian. This was done to obtain the Q-factor of the peaks. It is however better to fit all peaks at once instead of individually. The reason we did not do so is because there was spectra were not very clean. An example of this is the fact that there was crosstalk between the horizontal and vertical spectra. A more careful analysis could properly rotate the spectra to avoid this crosstalk. This might make it easier to do a single fit per spectra instead of 2 seperate fits.

\section*{Trapping at low pressures}
Qualitatively we found it to be very hard to trap the particle at low pressures ($<\qty{1}{\milli\bar}$). It is likely that more damping is needed to dissipate the energy of the particle or active feedback to trap the particle. This is something that we did not investigate further.
