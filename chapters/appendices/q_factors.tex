\chapter{Definition and conversion of Q-factors}
\label{app:q_factors}

In this appendix we consider a damped and driven harmonic oscillator described by the following equation of motion:
\begin{equation}
    \ddot{x} + \gamma \dot{x} + \omega_0^2 x = D(t)
\end{equation}
where $x$ is some amplitude in \unit{\meter}, $\gamma$ is the damping rate in \unit{\per\second}, $\omega_0 = 2\pi f_0$ is the resonance frequency in \unit{\radian\per\second} and $D(t)$ is the driving force. The Q-factor is a measure for the width (and height) of the resonance peak.

There are two common definitions of the Q-factor. The first is the \textit{bandwidth} definition, where the Q-factor is given by the ratio between the resonance frequency and the bandwidth of the resonance.
\begin{equation}
    Q_\text{B} = \frac{f_0}{\Delta f} = \frac{\omega_0}{\Delta \omega}
\end{equation}
Given a resonance peak in a spectrum, the bandwidth is defined as the width of the peak at half the maximum amplitude (FWHM). The FWHM should be determined by fitting a Lorentzian (Cauchy distribution) to the peak in the power spectrum.
\begin{equation}
    \mathcal{L}(f) \propto \frac{\Gamma}{(f - f_0)^2 + \Gamma^2}
\end{equation}
In this case $\Gamma$ is the scale of the peak and the FWHM is given by $2\Gamma$. Additionally the FWHM is equal to the damping rate $\gamma$. When looking at a peak in a `normal' Fourier spectrum, meaning you look at the amplitude of the signal, the magnitude of the peak is instead described by:
\begin{equation}
    A(f) \propto \sqrt{\frac{1}{(f - f_0)^2 + \gamma^2f^2}}
\end{equation}
The FWHM in this case is given by $\sqrt{2}\gamma$. This means that the Q-factor of a `normal' Fourier spectrum deviates from the Q-factor of a power spectrum by a factor of $\sqrt{2}$! Since it is most common to look at the power spectrum, we will use the power spectrum definition of the Q-factor.

Another way of defining the Q-factor is the \textit{energy} definition. The Q-factor is then given by the ratio between the energy stored in the oscillator and the energy dissipation.
\begin{equation}
    Q_\text{E} = 2 \pi \frac{\text{stored energy}}{\text{dissipated energy per cycle}}
\end{equation}
Or equivalently:
\begin{equation}
    Q_\text{E} = 2 \pi f_0 \frac{\text{stored energy}}{\text{dissipation}}
\end{equation}
The energy of a damped harmonic oscillator decays exponentially with time where the decay time is given by $1/\gamma$. Thus the Q-factor in terms of $\gamma$ is given by:
\begin{equation}
    Q_\text{E} = 2 \pi f_0 \frac{E(t)}{\tau E(t)} = 2 \pi f_0 \tau_\text{energy} = \frac{2\pi f_0}{\gamma}
\end{equation}
Since we often observe a decaying amplitude in the time domain, to convert this decay time to the energy domain we can use $\tau_\text{energy} = \frac{1}{2} \tau_\text{amplitude}$. To convert between any Q-factor we assert that they must result in the same damping rate $\gamma$. As this is the physical quantity that underlies the system. As a reference we summarize the conversion factors in the table below.

\begin{table}[h]
    \centering
    \begin{tabular}{ccc}
        \toprule
        \textbf{Observation} & $\bm{Q_B}$ & $\bm{Q_E}$ \\
        \midrule
        FWHM in PSD ($\Delta f$) & $\frac{f_0}{\Delta f}$ & $2\pi \frac{f_0}{\Delta f}$ \\
        FWHM in FFT ($\Delta f$) & $\sqrt{2} \frac{f_0}{\Delta f}$ & $2\pi \sqrt{2} \frac{f_0}{\Delta f}$ \\
        Decay in energy ($\tau_\text{energy}$) & $f_0 \tau_\text{energy}$ & $2 \pi f_0 \tau_\text{energy}$ \\
        Decay in amplitude ($\tau_\text{amplitude}$) & $\frac{1}{2} f_0 \tau_\text{amplitude}$ & $\pi f_0 \tau_\text{amplitude}$ \\
        \bottomrule
    \end{tabular}
\end{table}
