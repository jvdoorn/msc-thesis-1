\chapter{Theory}
\label{chap:theory}

\section{Magnetic levitation}
\label{sec:magnetic_levitation}
Obtaining and then maintaining stable magnetic levitation requires three fields: a static homogenous field to align the magnet ($\vec{B_0}$); an alternating field containing the saddle point ($\vec{B_1}$); and a gradient field to counter the gravitational offset ($\vec{B_2}$). This follows from simulations and our previous experience \cite{perdriat,mart,eli}.

The homogeneous field is given in \autoref{eq:homogeneous_field}. It is oriented in the $z$-axis with magnitude $B_0$. Homogenous fields can (locally) be created using a Helmholtz coil.
\begin{equation}
    \vec{B_0}(\vec{r}, t) = \vec{B_0} = B_0 \zhat
    \label{eq:homogeneous_field}
\end{equation}

Rotating a saddle point sufficiently fast effectively creates a (local) minimum. The field is given in \autoref{eq:saddle_point}. $B_1''$ is the curvature of $\vec{B_1}$ and $\Omega / 2\pi$ the frequency of the oscillation. The field can be created using two loops of wire with a current in opposite directions in the same plane, a so called magnetic Paul Trap. These loops are the main focus of this thesis. Given the radius of the inner loop $r_1$ we can express the curvature as $B_1'' = -\frac{9}{16}\mu_0i_1/r_1^3$ where $i_1$ is the current through the loop\footnote{This uses the requirement that $i_1/i_2 = -r_1/r_2$ and that $i_1$ and $i_2$ oppose each other (clockwise and anti-clockwise).}.
\begin{equation}
    \vec{B_1}(\vec{r}, t) = \frac{B_1''}{2} \begin{pmatrix}
        -xz \\
        +yz \\
        z^2 - \frac{1}{2}(x^2 + y^2)
    \end{pmatrix} \cos(\Omega t)
    \label{eq:saddle_point}
\end{equation}

Finally we have the gradient field, given in \autoref{eq:gradient_field}. The gradient can be expressed as $B_2' = mg/\mu$ where $m$ is the mass of the levitated particle, $g$ the gravitational acceleration and $\mu$ the magnetic moment of the particle. The gradient field can be created using ...?
\begin{equation}
    \vec{B_2}(\vec{r}, t) = B_2' \begin{pmatrix}
        -x / 2 \\
        -y / 2 \\
        z
    \end{pmatrix}
    \label{eq:gradient_field}
\end{equation}
