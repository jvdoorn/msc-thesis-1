\chapter{Method}
\label{chap:method}

\section{Sample fabrication}
\label{sec:sample-fabrication}
A \qty{9}{\mm} by \qty{5}{\mm} undoped \ce{Si} wafer of \qty{500}{\um} thick is used as a substrate. The substrate is spincoated at \qty{2000}{\rpm} with positive resist AR-P 662.06 and baked at \qty{150}{\celsius} for \qty{3}{\min}. This step is then repeated in order to coat 2 layers in total resulting in a total thickness of \qty{1}{\um}. A lithography step is performed using the Raith 100 EBPG exposing the resist to \qty{400}{\micro\coulomb\per\square\cm}. The resist is developed using a 1:3 mixture of MIBK and isopropanol for \qty{45}{\s}, the development is stopped using isopropanol. The Z-407 sputtering machine deposits a \qty{5}{\nm} \ce{MoGe} sticking layer followed by a \qty{500}{\nm} \ce{Au} layer. The lift-off is performed in acetone.

Inside of the inner loop of the coil, we mill a hole of roughly \qty{30}{\um} deep with a diameter of \qty{40}{\um} in the \ce{Si} substrate using a \ce{Ga+} focussed ion beam (Aquilos 2 Cryo-FIB). A similar hole is also made in a microscope coverslip. Using a micromanipulator a \ce{NdFeB} particle of roughly \qty{30}{\um} is placed inside the \ce{Si} hole\footnote{The easiest way to do so is by sticking the particle to the bottom of the needle and then scraping the particle off on the sides of the \ce{Si} hole.} and the coverslip is placed on top of the \ce{Si} substrate. The holes are carefully aligned by moving the coverslip using a micromanipulator. The coverslip is then glued to the \ce{Si} substrate using an epoxy.

A summary of the dimensions of the sample is shown in \autoref{tab:sample-dimensions}. For a schematic illustration of the sample see TODO.
\begin{table}
    \centering
    \begin{tabular}{lcc}
        \toprule
        \textbf{Parameter} & \textbf{Symbol} & \textbf{Value} \\
        \midrule
        Track thickness & $d$ & \qty{500}{\nm} \\
        Inner track width & $w$ & \qty{50}{\um} \\
        Inner track width & $w$ & \qty{100}{\um} \\
        Inner loop radius & $r_1$ & \qty{???}{\um} \\
        Outer loop radius & $r_2$ & \qty{???}{\um} \\
        \ce{Si}/Glass hole diameter & & \qty{100}{\um} \\
        \ce{Si}/Glass hole depth & & \qty{15}{\um} \\
        \bottomrule
    \end{tabular}
    \caption{Dimensions of the sample. The radii of the inner and outer loops are the radii at the center of the track.}
    \label{tab:sample-dimensions}
\end{table}

\section{Experimental setup}
The sample is mounted on a printed circuit board (PCB) with a thick (???) copper baseplate. The top of the sample is exactly flush with the top of the PCB. Additionally the pads for the wirebonds on the sample and the PCB are aligned and directly next to each other. This allows for very short wirebonds to be used. This significantly reduces the resistance of the wirebonds and thus the maximum current that can be applied to the sample. The wirebonds are made of \qty{???}{\um} thick \ce{Pt}.

The PCB is then placed inbetween two Helmholtz coils with a diameter of \qty{???}{\um} and ??? windings. They Helmholtz coils provide the uniform magnetic field to align the sample and the gradient magnetic field to control the vertical position of the particle. A dc current on the order of \qty{1}{\ampere} is sent through the coils by two power supplies (Tenma ???).

A microscope objective (???) with a $20\times$ magnification is used to provide a visual image of the sample. Additionally a laser (???) is coupled in. The reflection of the laser is then imaged on a photodiode (???). The photodiode is connected to a lock-in amplifier (???) allowing for the detection of the motion of the particle.
