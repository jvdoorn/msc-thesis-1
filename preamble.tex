% !TEX root = thesis.tex

% Fixes things like bold font.
\usepackage[tuenc]{fontspec}

% Recommended by babel.
\usepackage{csquotes}

% Use siunitx to write out units and quantities, use special formatting for the units.
\usepackage{siunitx}
\sisetup{separate-uncertainty = true, round-mode = uncertainty, multi-part-units = single, inter-unit-product = \ensuremath { { } \cdot { } }, range-units = single, list-units = single, exponent-product=\cdot}
\robustify{\dots}
\sisetup{input-digits = 0123456789\dots?}
\DeclareSIUnit\fluxquantum{\text{\ensuremath{\Phi_0}}}
\DeclareSIUnit\rpm{\text{rpm}}
\DeclareSIUnit\fps{\text{fps}}

% Setup bibliography (instead of relying on the way lion-msc does it using natbib).
\usepackage[backend=biber, style=phys, biblabel=brackets]{biblatex}
\addbibresource{sources.bib}

% Let footnotes use numbers instead of symbols.
\renewcommand{\thefootnote}{\arabic{footnote}}

% For double closed integrals among others.
\usepackage{esint}
% Diameter symbol.
\usepackage{wasysym}

% For chemical formulas.
\usepackage[version=4]{mhchem}

% Nicer tables.
\usepackage{booktabs}
\usepackage{tabularx}

% Easier to use lists.
\usepackage{enumitem}

% Used for including PGF files. Use the external package to avoid memory issues and increase compilation speed.
\usepackage{import}
\usepackage{pgfplots}
\pgfplotsset{compat=1.18}
\usepgfplotslibrary{external}
\tikzexternalize

% Used for subfigures/captions.
\usepackage{subcaption, sidecap}
% Align the caption of figures at the baseline and tables at the top.
\sidecaptionvpos{figure}{c}
\sidecaptionvpos{table}{t}

% Used for debugging such as printing the value of \textwidth (\printinunitsof{in}\prntlen{\textwidth})
\usepackage{layouts}

% Use whitespace instead of indent for new paragraphs.
\usepackage[parfill]{parskip}

% Used for properly typesetting bold symbols in math mode.
\usepackage{bm}

% Redefine \vec to use bold font instead of an arrow.
\renewcommand{\vec}[1]{\bm{#1}}
% Additionally define shorthands for the unit vectors in the x, y and z directions.
\newcommand{\xhat}{\vec{e_x}}
\newcommand{\yhat}{\vec{e_y}}
\newcommand{\zhat}{\vec{e_z}}

% Used for the \abs command.
\newcommand{\abs}[1]{\left\lvert#1\right\rvert}

% Used in plots to clarify the x-, y- and z-directions.
\definecolor{x_axis_color}{HTML}{EE6677}
\definecolor{y_axis_color}{HTML}{228833}
\definecolor{z_axis_color}{HTML}{4477AA}
\def\xmode{\textcolor{x_axis_color}{x}-mode}
\def\ymode{\textcolor{y_axis_color}{y}-mode}
\def\zmode{\textcolor{z_axis_color}{z}-mode}
